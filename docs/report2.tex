\documentclass[12pt,a4paper]{article}
\usepackage[english]{babel}
\usepackage{listings, graphicx}
\lstset{language=Python}

\begin{document}

\begin{titlepage}

\begin{center}

\includegraphics[scale=0.3]{logo_cidadania_2.eps}\\

\textsc{\Huge e-cidadania}\\
\large Citizen participation platform based on web services\\[0.3cm]

\textsc{\Large strategy report}\\[1cm]
{\normalsize \today}

\end{center}

\begin{minipage}{0.5\textwidth}
\begin{flushleft}
\emph{Author:}\\
Oscar \textsc{Carballal Prego} \texttt{<info@oscarcp.com>}
\end{flushleft}
\end{minipage}
\begin{minipage}{0.5\textwidth}
\begin{flushright}
\emph{Company:}\\
Cidadania Sociedad Cooperativa\\
\texttt{<cidadania@cidadania.coop>}
\end{flushright}
\end{minipage}
\end{titlepage}

\newpage

\begin{abstract}
This paper represents the work for a citizen participation project through the web (E-CIDADANIA) which will allow the users participate with their proposals and debate the proposals before they get to another place.

Blah blah blah blah blahblah
\end{abstract}
\newpage

\tableofcontents
\newpage

\section{Introduction}
Section content

\section{Architetural model}
In this citizen participation platform the users will have access to the system in a lot of ways: debates, proposals, wiki, documents, votes, etc. The applications can be configured to do various ways of authentication, also the security leves vary depending the scenario. Some parts of the system are accesible as an anonymous user.

\begin{itemize}
\item Flexibility configuring the enviroment. The environment must be adaptable for every place.
\item Multiple sites per instance. The platform must support multime sites per server instance to avoid saturation.
\item Multiple authentication methods. LDAP, DB users, OpenID.
\item Modular. The system must be modular so at any time it can be updated with new applications or enhancements.

The architecture proposed for the application is:

\end{itemize}

\section{Definition of e-cidadania platform}
asdfadfasd

\end{document}